\chapter{Introduction} \label{ch:introduction}

%\vvase{} was designed as a tool that could be useful throughout the process of designing and testing a vehicle.  The vision was a single product that could perform kinematic analysis, dynamic simulation, process tire data and replay recorded data.

\vvase{} is a tool for analyzing suspension kinematics.  Given the locations of the suspension linkages and the kinematic state of the vehicle (pitch, roll, heave and steer), many output values are computed.  These outputs are key indicators of suspension performance and vehicle handling qualities.

There are other tools available that perform similar functions, but \vvase{} was designed with two unique features that make it stand out.

First, \vvase{} was designed to ease the process of evaluating differences between design options.  Always visible is a panel that displays the differences in kinematic outputs for all open car files for the specified kinematic state.  Additionally, iteration files can be used to compare kinematic outputs across a range of kinematic states - simultaneously for as many car files as desired.  The iterations and panel displaying the kinematic outputs update continuously as suspension hardpoints are varied.

Second, \vvase{} provides a unique facility for performing complex optimizations.  These optimizations are based on a genetic search algorithm that allows the user complete control over the optimization process.  Genetic search algorithms allow broad coverage of solution spaces and can help avoid getting trapped at local minima, even for solution spaces which are non-linear or have steep gradients.  In suspension design, moving a single hardpoint can affect many output parameters.  An optimization tool that balances multiple constraints can be invaluable.

Currently, \vvase{} is limited to cars employing a double A-arm suspension at both ends.  It supports several spring/damper attachment methods, U-bar and T-bar anti-roll bars, asymmetric suspensions and several other configuration options that allow virtually any double A-arm suspension to be modeled.
